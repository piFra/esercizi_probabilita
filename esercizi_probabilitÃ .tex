%%%%%%%%%%%%%%%%%%%%%%%%%%%%%%%%%%%%%%%%%
% Short Sectioned Assignment
% LaTeX Template
% Version 1.0 (5/5/12)
%
% This template has been downloaded from:
% http://www.LaTeXTemplates.com
%
% Original author:
% Frits Wenneker (http://www.howtotex.com)
%
% License:
% CC BY-NC-SA 3.0 (http://creativecommons.org/licenses/by-nc-sa/3.0/)
%
%%%%%%%%%%%%%%%%%%%%%%%%%%%%%%%%%%%%%%%%%

%----------------------------------------------------------------------------------------
%	PACKAGES AND OTHER DOCUMENT CONFIGURATIONS
%----------------------------------------------------------------------------------------

\documentclass[paper=a4, fontsize=11pt]{scrartcl} % A4 paper and 11pt font size

%\usepackage[T1]{fontenc} % Use 8-bit encoding that has 256 glyphs
\usepackage{fourier} % Use the Adobe Utopia font for the document - comment this line to return to the LaTeX default
\usepackage[italian]{babel} % Italian language/hyphenation
\usepackage[latin1]{inputenc} % For some italian specific letters(accents)
\usepackage[fleqn]{amsmath} % Math package
\usepackage{amsfonts,amsthm} % Math packages

\usepackage{sectsty} % Allows customizing section commands
\allsectionsfont{\centering \normalfont\scshape} % Make all sections centered, the default font and small caps

\usepackage{fancyhdr} % Custom headers and footers
\pagestyle{fancyplain} % Makes all pages in the document conform to the custom headers and footers
\fancyhead{} % No page header - if you want one, create it in the same way as the footers below
\fancyfoot[L]{} % Empty left footer
\fancyfoot[C]{} % Empty center footer
\fancyfoot[R]{\thepage} % Page numbering for right footer
\renewcommand{\headrulewidth}{0pt} % Remove header underlines
\renewcommand{\footrulewidth}{0pt} % Remove footer underlines
\setlength{\headheight}{13.6pt} % Customize the height of the header

\numberwithin{equation}{section} % Number equations within sections (i.e. 1.1, 1.2, 2.1, 2.2 instead of 1, 2, 3, 4)
\numberwithin{figure}{section} % Number figures within sections (i.e. 1.1, 1.2, 2.1, 2.2 instead of 1, 2, 3, 4)
\numberwithin{table}{section} % Number tables within sections (i.e. 1.1, 1.2, 2.1, 2.2 instead of 1, 2, 3, 4)

\setlength\parindent{0pt} % Removes all indentation from paragraphs - comment this line for an assignment with lots of text

%----------------------------------------------------------------------------------------
%	TITLE SECTION
%----------------------------------------------------------------------------------------

\newcommand{\horrule}[1]{\rule{\linewidth}{#1}} % Create horizontal rule command with 1 argument of height

\title{	
\normalfont \normalsize 
\textsc{Universita' degli studi di Catania} \\ [25pt] % Your university, school and/or department name(s)
\horrule{0.5pt} \\[0.4cm] % Thin top horizontal rule
\huge Esercizi calcolo probabilistico \\ % The assignment title
\horrule{2pt} \\[0.5cm] % Thick bottom horizontal rule
}

\author{Francesco Pistritto} % Your name

\date{\normalsize\today} % Today's date or a custom date

\begin{document}

\maketitle % Print the title

%----------------------------------------------------------------------------------------
%	PROBLEM 1
%----------------------------------------------------------------------------------------

\section*{Problema 1}

\textbf{
Si considerino 2 urne, A e B. L'urna A contiene 2 palline bianche e 2 palline nere. L'urna B contiene 3 palline bianche e 2 nere. Si estrae una pallina da A e, senza vederla, la si inserisce dentro B. Successivamente si estrae da B, senza reinserimento, una pallina che risulta essere bianca. Si determini la probabilit\'a che la pallina trasferita da A a B sia stata bianca, sapendo che la pallina estratta da B \'e bianca.\\
Inoltre si vuole estrarre un'ulteriore pallina da B. Si calcoli la probabilit\'a che tale pallina sia bianca.
}

%------------------------------------------------

\subsection*{Soluzione}
Per la risoluzione del problema \'e conviene utilizzare il teorema di Bayes, in quanto il quesito \'e una probabilit\'a condizionata.\\
Prima del passaggio della pallina da A a B, le urne contenevano:\\
\texttt{A: 2B 2N}\\
\texttt{B: 3B 2N}\\
Definiamo:
\begin{align}
X_{1} &= passaggio\; di\; una\; pallina\; bianca\; da\; A\; a\; B\\
X_{2} &= estrazione\; di\; una\; pallina\; bianca\; da\; B\\
X_{3} &= passaggio\; di\; una\; pallina\; nera\; da\; A\; a\; B\\
P(X_1) &= probabilit\'a\; di\; estrarre\; una\; bianca\; da\; A\\
P(X_3) &= probabilit\'a\; di\; estrarre\; una\; nera\; da\; A
\end{align}
So che $P(X_1)=P(X_3)=\frac{1}{2}$ poich\'e se ho 4 palline uguali a due a due, la probabilit\'a di prenderne una coppia uguale \'e $\frac{1}{2}$.\\
Per risolvere il problema impostiamo l'equazione utilizzando il teorema di Bayes:
\begin{align}
P(X_{1} | X_{2}) = \frac{P(X_{2} | X_{1}) \cdot P(X_{1})}{P(X_{2})}
\end{align}
Per trovare $P(X_{2})$ dobbiamo utilizzare il teorema della probabilit\'a totale, impostandolo come segue:
$$
P(X_{2}) = P(X_2 | X_1) \cdot P(X_1) + P(X_2 | X_3) \cdot P(X_3) = \frac{4}{6} \cdot \frac{2}{4} + \frac{3}{6} \cdot \frac{2}{4} = \frac{7}{12}
$$
Stiamo descrivendo la probabilit\'a di estrarre una pallina bianca da B come la somma delle probabilit\'a di: estrarre una pallina bianca da B dato per certo che \'e stata passata da A a B una pallina bianca, per la probabilit\'a che ci\'o sia effettivamente avvenuto, pi\'u la probabilit\'a di estrarre una pallina bianca da B dato per certo che sia stata passata da A a B una pallina nera, per la probabilit\'a che ci\'o sia effettivamente avvenuto.\\
Adesso sostituiamo nella formula e otteniamo:
$$
P(X_{1} | X_{2}) = \frac{\frac{4}{6} \cdot \frac{2}{4}}{\frac{7}{12}} = \frac{7}{12} = 0.5714
$$
%------------------------------------------------

%----------------------------------------------------------------------------------------
%	PROBLEM 2
%----------------------------------------------------------------------------------------
\section*{Problema 2}

\textbf{
Si considerino 10 urne, identiche in apparenza, di cui nove contengono due palline bianche e due nere e una contiene cinque palline bianche e una nera. Scelta un'urna a caso, si estrae una pallina che risulta essere bianca. Quale \'e la probabilit\'a che la pallina sia stata estratta dall'urna contenente 5 palline bianche?
}

\subsection*{Soluzione}
Per risolvere questo problema \'e conveniente utilizzare il teorema di Bayes in quanto il quesito richiede una probabilit\'a condizionata.
Definiamo:
\begin{align}
U_2 &= Urna\; contenente\; 2\; palline\; bianche\\
U_5 &= Urna\; contenente\; 5\; palline\; bianche\\
X_1 &= estrazione\; di\; una\; pallina\; bianca\; da\; U_2\\
X_2 &= estrazione\; di\; una\; pallina\; bianca\; da\; U_5\\
X_3 &= estrazione\; di\; una\; pallina\; bianca\\
P(X_1) &= probabilit\'a\; di\; scegliere\; un'urna\; di\; tipo\; 2\\
P(X_2) &= probabilit\'a\; di\; scegliere\; un'urna\; di\; tipo\; 5
\end{align}
Il problma ci chiede di trovare la probabilit\'a condizionata: $P(X_2 | X_3)$ che con il teorema di Bayes possiamo esprimere come:
$$
P(X_2 | X_3) = \frac{P(X_3 | X_2) \cdot P(X_2)}{P(X_3)}
$$
dove per trovare $P(X_3)$ utilizziamo il teorema sulla probabilit\'a totale, sapendo che: $ P(X_1) = \frac{9}{10} \\ P(X_2) = \frac{1}{10}$
$$
P(X_3) = P(X_3 | X_1) \cdot P(X_1) + P(X_3 | X_2) \cdot P(X_2) 
$$
Stiamo esprimendo la probabilit\'a di estrarre una pallina bianca da tutte le urne come la somma delle probabilit\'a di: estrarre una pallina bianca dall'urna dato per certo che l'ho estratta da un'urna di tipo 2 per la probabilit\'a che abbia effettivamente estratto la pallina da un'urna di tipo 2, pi\'u la probabilit\'a di estrarre una pallina bianca dall'urna dato per certo che l'ho estratta da un'urna di tipo 5 per la probabilit\'a che abbia effettivamente estratto la pallina da un'urna di tipo 5\\
Sostituendo otteniamo:
$$
P(X_3) = \frac{2}{4} \cdot \frac{9}{10} + \frac{5}{6} \cdot \frac{1}{10} = \frac{32}{60}
$$
Adesso utilizziamo la formula di Bayes con i valori ottenuti:
$$
P(X_2 | X_3) = \frac{\frac{5}{6} \cdot \frac{1}{10}}{\frac{32}{60}} = \frac{5}{32} = 0.1562
$$
%----------------------------------------------------------------------------------------

%----------------------------------------------------------------------------------------
%	PROBLEM 3
%----------------------------------------------------------------------------------------
\section*{Problema 3}

\textbf{
La probabilit\'a che un giocatore di basket faccia canestro al tiro libero \'e 0.7. Assumendo che in una partita ci siano 15 tiri liberi, calcolare: \\
\begin{itemize}
\item La probabilit\'a che il giocatore metta a segno tutti i tiri liberi
\item La probabilit\'a che il giocatore metta a segno 14 tiri liberi
\end{itemize}
}

\subsection*{Soluzione}
Il problema si presta ad essere risolto utilizzando la probabilit\'a definita secondo Bernoulli, attraverso l'utilizzo dei coefficienti binomiali $\binom{n}{k}$ dove \texttt{n} \'e il numero di tentativi e \texttt{k} il numero di successi desiderati. La formula completa da utilizzare \'e: $\binom{n}{k}p^{k}q^{n-k}$.\\
Definiamo:
\begin{align}
X_1 &= probabilit\'a\; di\; ottenere\; 15\; successi\\
X_2 &= probabilit\'a\; di\; ottenere\; 14\; successi\\
n &= 15\\
k_1 &= 15\\
k_2 &= 14\\
p &= 0.7\\
q &= 1-p = 0.3
\end{align}
Dove $k_1$ \'e il numero di successi desiderati al primo punto del problema, mentre $k_2$ \'e il numero di successi desiderati al secondo punto del problema. \\
Sostituendo i valori per risolvere il primo punto otteniamo:
$$
P(X_1) = \binom{15}{15} \cdot 0.7^{15} \cdot 0.3^{0} = 0.0047
$$
Mentre se sostituiamo i valori per ottenere la soluzione del secondo problema otteniamo:
$$
P(X_2) = \binom{15}{14} \cdot 0.7^{14} \cdot 0.3^{1} = 0.0305
$$
%----------------------------------------------------------------------------------------

%----------------------------------------------------------------------------------------
%	PROBLEM 3
%----------------------------------------------------------------------------------------
\section*{Problema 4}
\textbf
{
I una fabbrica di cioccolato, viene impiegato un sistema automatico di riempimento delle confezioni di cioccolato fuso per la confezione delle torte. Supponendo che la quantit\'a X di cioccolato fuso che viene messa in ciascuna confezione dal macchinario A possa essere rappresentata come una variabile aleatoria normale con media $\mu = 0.5lt$ e varianza $\sigma ^2$, determinare:
\begin{enumerate}
\item Il valore di $\sigma$ affinch\'e la percentuale di confezioni riempite con questo macchinario che contengono pi\'u di 1lt di prodotto sia inferiore al 10\%.
\end{enumerate}
Supponendo poi che che il 25\% delle confezioni venga riempito con il macchinario A e il restante 75\% con il macchinario B e sapendo che la probabilit\'a che il flacone contenga meno di 0.5lt di cioccolato fuso \'e uguale al 20\% quando viene usato il macchinario B, determinare:
\begin{enumerate}
\item[2] La probabilit\'a che, presa a caso una confezione di cioccolato fuso, il suo contenuto sia inferiore a 0.5lt.
\item[3] La probabilit\'a che, presa a caso una confezione di cioccolato fuso che contiene meno di 0.5lt di cioccolato fuso, questa sia stata riempita dal macchinario A.
\end{enumerate}
}
\subsection{Soluzione}
Il primo punto del problema si presta ad essere risolto utilizzando le formule della variabile normale standard, ed in particolare la funzione $\Phi(x)$. Definiamo:
\begin{align}
X &= quantit\'a\; di\; cioccolato\; per\; confezione\; se\; usiamo\; A\\
\lambda &= \frac{X - \mu_A}{\sigma_A} \to variabile\; normalizzata\\
P(X > 1) &= probabilit\'a\; che\; usando\; A\; si\; abbiano\; pi\'u\; di\; 1lt\; di\; cioccolato\;
\end{align}
Il primo punto chiede di trovare: $P(X > 1) < 0.1$ da cui, effettuando la sostituzione:\\ $P(X > 1) = 1 - P(X \leq 1)$ otteniamo 
$$
1 - P(X \leq 1) < 0.1 \to P(X \leq 1) > 0.9
$$
Adesso calcoliamo $\lambda$ come $\lambda = \frac{1 - 0.5}{\sigma_A} = \frac{0.5}{\sigma_A}$. Notiamo che $\sigma_A$ \'e una quantit\'a positiva, in quanto \'e una radice ($\sigma_A = \sqrt{\sigma_A^2}$).\\ Dunque sapendo che $\lambda = \frac{0.5}{\sigma_A} > 0$, possiamo calcolare $\Phi(\lambda)$ come $\Phi(\lambda) = 0.5 + Tab(\lambda)$ ed imporre che $\Phi(\lambda) > 0.9$.\\
Sostituendo a $\Phi(\lambda)$ il valore $0.5 + Tab(\lambda)$ otteniamo $0.5 + Tab(\lambda) > 0.9$ da cui $Tab(\lambda) \geq 0.4$.\\
Andiamo a leggere in corrispondenza della tabella della $\Phi(x)$ quando troviamo un valore di $x$ che rende $\Phi(x) > 0.4$. Tale valore \'e circa $1.29$, da cui ricaviamo che $\frac{0.5}{\sigma_A} \geq 1.29 \to \sigma_A \geq 0.384$.\\
Scegliamo $\sigma_A = 0.384$ poich\'e \'e il pi\'u piccolo valore per cui la disuguaglianza \'e verificata.\\
\subsection{Soluzione}
Per la soluzione del secondo punto del problema definiamo ancora:
\begin{align}
P(A) = Probabilit\'a\; che\; una\; confezione\; sia\; riempita\; con\; A\\
P(B) = Probabilit\'a\; che\; una\; confezione\; sia\; riempita\; con\; B\\
\end{align}
Sappiamo che $P(A) = 0.25$ e che $P(B) = 0.75$, quindi per calcolare $P(X < 0.5)$ possiamo usare il teorema della probabilit\'a totale come:
$$
P(X < 0.5) = P(X < 0.5 | B) \cdot P(B) + P(X < 0.5 | A) \cdot P(A)
$$
\textbf{\texttt{N.B.}} $P(X < 0.5 | A) = \frac{1}{2}$ poich\'e $\mu_A = \frac{1}{2}$, dunque, dato che l'area della curva vale 1, basta perndere met\'a della curva per avere un'area pari a $\frac{1}{2}$.
Sostituiamo i valori e otteniamo:
$$
P(X < 0.5) = \frac{2}{10} \cdot \frac{3}{4} + \frac{1}{2} \cdot \frac{1}{4} = \frac{11}{40} = 0.2750
$$
\subsection{Soluzione}
Possiamo adesso risolvere il terzo punto del problema utilizzando il teorema di Bayes come:
$$
P(A | X < 0.5) = \frac{P(X < 0.5 | A) \cdot P(A)}{P(X < 0.5)} = \frac{\frac{1}{2} \cdot \frac{1}{4}}{\frac{11}{40}} = \frac{5}{11} = 0.4545
$$
%----------------------------------------------------------------------------------------


\end{document}
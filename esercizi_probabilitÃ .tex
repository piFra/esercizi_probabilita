%%%%%%%%%%%%%%%%%%%%%%%%%%%%%%%%%%%%%%%%%
% Short Sectioned Assignment
% LaTeX Template
% Version 1.0 (5/5/12)
%
% This template has been downloaded from:
% http://www.LaTeXTemplates.com
%
% Original author:
% Frits Wenneker (http://www.howtotex.com)
%
% License:
% CC BY-NC-SA 3.0 (http://creativecommons.org/licenses/by-nc-sa/3.0/)
%
%%%%%%%%%%%%%%%%%%%%%%%%%%%%%%%%%%%%%%%%%

%----------------------------------------------------------------------------------------
%	PACKAGES AND OTHER DOCUMENT CONFIGURATIONS
%----------------------------------------------------------------------------------------

\documentclass[paper=a4, fontsize=11pt]{scrartcl} % A4 paper and 11pt font size

%\usepackage[T1]{fontenc} % Use 8-bit encoding that has 256 glyphs
\usepackage{fourier} % Use the Adobe Utopia font for the document - comment this line to return to the LaTeX default
\usepackage[italian]{babel} % Italian language/hyphenation
\usepackage[latin1]{inputenc} % For some italian specific letters(accents)
\usepackage{amsmath,amsfonts,amsthm} % Math packages

\usepackage{sectsty} % Allows customizing section commands
\allsectionsfont{\centering \normalfont\scshape} % Make all sections centered, the default font and small caps

\usepackage{fancyhdr} % Custom headers and footers
\pagestyle{fancyplain} % Makes all pages in the document conform to the custom headers and footers
\fancyhead{} % No page header - if you want one, create it in the same way as the footers below
\fancyfoot[L]{} % Empty left footer
\fancyfoot[C]{} % Empty center footer
\fancyfoot[R]{\thepage} % Page numbering for right footer
\renewcommand{\headrulewidth}{0pt} % Remove header underlines
\renewcommand{\footrulewidth}{0pt} % Remove footer underlines
\setlength{\headheight}{13.6pt} % Customize the height of the header

\numberwithin{equation}{section} % Number equations within sections (i.e. 1.1, 1.2, 2.1, 2.2 instead of 1, 2, 3, 4)
\numberwithin{figure}{section} % Number figures within sections (i.e. 1.1, 1.2, 2.1, 2.2 instead of 1, 2, 3, 4)
\numberwithin{table}{section} % Number tables within sections (i.e. 1.1, 1.2, 2.1, 2.2 instead of 1, 2, 3, 4)

\setlength\parindent{0pt} % Removes all indentation from paragraphs - comment this line for an assignment with lots of text

%----------------------------------------------------------------------------------------
%	TITLE SECTION
%----------------------------------------------------------------------------------------

\newcommand{\horrule}[1]{\rule{\linewidth}{#1}} % Create horizontal rule command with 1 argument of height

\title{	
\normalfont \normalsize 
\textsc{Universita' degli studi di Catania} \\ [25pt] % Your university, school and/or department name(s)
\horrule{0.5pt} \\[0.4cm] % Thin top horizontal rule
\huge Esercizi calcolo probabilistico \\ % The assignment title
\horrule{2pt} \\[0.5cm] % Thick bottom horizontal rule
}

\author{Francesco Pistritto} % Your name

\date{\normalsize\today} % Today's date or a custom date

\begin{document}

\maketitle % Print the title

%----------------------------------------------------------------------------------------
%	PROBLEM 1
%----------------------------------------------------------------------------------------

\section*{Problema 1}

\textbf{Si considerino 2 urne, A e B. L'urna A contiene 2 palline bianche e 2 palline nere. L'urna B contiene 3 palline bianche e 2 nere. Si estrae una pallina da A e, senza vederla, la si inserisce dentro B. Successivamente si estrae da B, senza reinserimento, una pallina che risulta essere bianca. Si determini la probabilit\'a che la pallina trasferita da A a B sia stata bianca, sapendo che la pallina estratta da B \'e bianca.\\
Inoltre si vuole estrarre un'ulteriore pallina da B. Si calcoli la probabilit\'a che tale pallina sia bianca.}

%------------------------------------------------

\subsection*{Soluzione}
Per la risoluzione del problema \'e conviene utilizzare il teorema di Bayes, in quanto il quesito \'e una probabilit\'a condizionata.\\
Prima del passaggio della pallina da A a B, le urne contenevano:\\
\texttt{A: 2B 2N}\\
\texttt{B: 3B 2N}\\
Definiamo:
\begin{align}
X_{1} = passaggio\; di\; una\; pallina\; bianca\; da\; A\; a\; B\\
X_{2} = estrazione\; di\; una\; pallina\; bianca\; da\; B\\
X_{3} = passaggio\; di\; una\; pallina\; nera\; da\; A\; a\; B\\
P(X_1) = probabilit\'a\; di\; estrarre\; una\; bianca\; da\; A\\
P(X_3) = probabilit\'a\; di\; estrarre\; una\; nera\; da\; A
\end{align}
So che $P(X_1)=P(X_3)=\frac{1}{2}$ poich\'e se ho 4 palline uguali a due a due, la probabilit\'a di prenderne una coppia uguale \'e $\frac{1}{2}$.\\
Per risolvere il problema impostiamo l'equazione utilizzando il teorema di Bayes:
\begin{align}
P(X_{1} | X_{2}) = \frac{P(X_{2} | X_{1}) \cdot P(X_{1})}{P(X_{2})}
\end{align}
Per trovare $P(X_{2})$ dobbiamo utilizzare il teorema della probabilit\'a totale, impostandolo come segue:
$$
P(X_{2}) = P(X_2 | X_1) \cdot P(X_1) + P(X_2 | X_3) \cdot P(X_3) = \frac{4}{6} \cdot \frac{2}{4} + \frac{3}{6} \cdot \frac{2}{4} = \frac{7}{12}
$$
Stiamo descrivendo la probabilit\'a di estrarre una pallina bianca da B come la somma delle probabilit\'a di: estrarre una pallina bianca da B dato per certo che \'e stata passata da A a B una pallina bianca, per la probabilit\'a che ci\'o sia effettivamente avvenuto, pi\'u la probabilit\'a di estrarre una pallina bianca da B dato per certo che sia stata passata da A a B una pallina nera, per la probabilit\'a che ci\'o sia effettivamente avvenuto.\\
Adesso sostituiamo nella formula e otteniamo:
$$
P(X_{1} | X_{2}) = \frac{\frac{4}{6} \cdot \frac{2}{4}}{\frac{7}{12}} = \frac{7}{12} = 0.5714
$$
%------------------------------------------------

%----------------------------------------------------------------------------------------
%	PROBLEM 2
%----------------------------------------------------------------------------------------
\section*{Problema 2}

\textbf{Si considerino 10 urne, identiche in apparenza, di cui nove contengono due palline bianche e due nere e una contiene cinque palline bianche e una nera. Scelta un'urna a caso, si estrae una pallina che risulta essere bianca. Quale \'e la probabilit\'a che la pallina sia stata estratta dall'urna contenente 5 palline bianche?}

\subsection*{Soluzione}
Per risolvere questo problema \'e conveniente utilizzare il teorema di Bayes in quanto il quesito richiede una probabilit\'a condizionata.
Definiamo:
\begin{align}
U_2 = Urna\; contenente\; 2\; palline\; bianche\\
U_5 = Urna\; contenente\; 5\; palline\; bianche\\
X_1 = estrazione\; di\; una\; pallina\; bianca\; da\; U_2\\
X_2 = estrazione\; di\; una\; pallina\; bianca\; da\; U_5\\
X_3 = estrazione\; di\; una\; pallina\; bianca
\end{align}
Il problma ci chiede di trovare la probabilit\'a condizionata: $P(X_2 | X_3)$
%----------------------------------------------------------------------------------------

\end{document}